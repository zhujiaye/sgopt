\renewcommand{\algorithmicrequire}{\textbf{Input:}}
\renewcommand{\algorithmicensure}{\textbf{Output:}}
\rhead{基于线段树优化的动态PPI网络匹配}
\chapter{基于线段树优化的动态PPI网络匹配}
%@@@@@@@@@@@@@@@@@@@@@@@@@@@@@@@@@@@@@@@@@@@@@
\section{动态PPI网络及动态PPI网络匹配}
由于本文要研究的问题是如何在动态PPI网络上作网络匹配,因此要先对动态PPI网络以及动态PPI网络匹配问题重新做出定义。

%@@@@@@@@@@@@@@@@@@@@@@@@@@@@@@@@@@@@@@@@@@@@@
\subsection{动态PPI网络定义}
一个动态PPI网络定义为四元组$G=(V,E,L,R)$,其中$V$是点集,$E$是边集,对于边集中的每一条边表示为$(u,v)$,其中$u$和$v$是在某一时刻具有相互作用的蛋白质点对。$L(u,v)$为边$(u,v)$处于活跃状态的开始时刻,$R(u,v)$为边$(u,v)$处于活跃状态的结束时刻,定义

\begin{equation}\label{myworkactivedefine}
Act_i(u,v)=\begin{cases}
1 & \text{  } L(u,v)\leq i\leq R(u,v), (u,v)\in E_2\\ 
0 & \text{  其他}  
\end{cases}
\end{equation}

用来表示边$(u,v)$在$i$时刻是否处于活跃状态。定义$Act_i(E)=\{(u,v):(u,v)\in E,Act_i(u,v)=1\}$表示$i$时刻下边集E中活跃的边组成的集合。

直观上看,一个动态PPI网络,就是在对应的静态PPI网络的基础上,加入了时间的元素,即每一条边$(u,v)$只会存在与某个连续的时间段$[L(u,v),R(u,v)]$内,而不是一直处于活跃状态。值得注意的是,在\cite{zhang2016method}中,动态PPI网络的定义与本文的定义稍有不同,本文为了简化问题,将边的活跃从原来的概率模型,转化成了0-1模型(即某一时刻,边$(u,v)$要么活跃,要么不活跃),并且假设每条边的活跃范围是一个连续的时间区间。对于多个时间区间的动态PPI网络定义也是可以的,本文的算法在这种情况下也是可以扩展的,简单起见,本文定义的时间区间个数限制在1个。

%@@@@@@@@@@@@@@@@@@@@@@@@@@@@@@@@@@@@@@@@@@@
\subsection{动态PPI网络构造}
本文主要用了\cite{zhang2016method}中的三-西格玛阈值方法,对静态PPI网络进行了动态化的转化,并且假设当边处于活跃状态的概率大于$0.5$时,边就是处于活跃状态的,否则就是不活跃状态。详细的动态PPI网络构造方法可以参见相关工作。

%@@@@@@@@@@@@@@@@@@@@@@@@@@@@@@@@@@@@@@@@@@
\subsection{动态PPI网络匹配}
本文研究的问题主要是一个静态PPI网络与一个动态PPI网络的匹配,两个动态PPI网络之间的匹配留作未来的研究工作。

有一个静态PPI网络$G_1=(V_1,E_1)$和一个动态PPI网络$G_2=(V_2,E_2,L,R)$,$|V_1|\leq |V_2|$定义动态PPI匹配$f:V_1\rightarrow V_2$为点集$V1$到点集$V2$的一个单射。

在源网络$G_1$中,如果边$(u,v)$在时刻$i$满足$Act_i(f(u),f(v))=1$,则称边$(u,v)$在时刻$i$是被保留(匹配)的。

定义$f_i(E_1)=|\{(u,v):(u,v)\in E_1, \text{且} (u,v)\text{在$i$时刻是被保留的}\}|$为源网络边集$E_1$在$i$时刻被保留的边的集合。

在动态PPI网络的环境下,所有的匹配衡量标准都需要重新定义,我们称这些衡量标准为动态匹配衡量标准(dynamic alignment quality measure)。分别有
\begin{equation}\label{myworkecdefine}
    EC(f)=\underset{i}{max}\frac{\left | f_i(E_1) \right |}{\left | E_1 \right |}
\end{equation}
\begin{equation}\label{myworkicsdefine}
    ICS(f)=\underset{i}{max}\frac{\left | f_i(E_1) \right |}{\left |Act_i(E_2(G_2[f(V_1)]))\right |}
\end{equation}
\begin{equation}\label{myworks3define}
S^{3}(f)=\underset{i}{max}\frac{\left | f_i(E_1) \right |}{\left | E_1 \right |+\left |Act_i(E_2(G_2[f(V_1)])) \right |-\left | f_i(E_1) \right |}
\end{equation}
\begin{equation}\label{myworktwecdefine}
    TWEC(f)=\frac{EC(f)+ICS(f)}{2}
\end{equation}
可以看到,在动态PPI网络的环境下,所有的衡量标准,都得到了动态化的定义,其本质,就是在匹配$f$下,将动态PPI网络看成若干个静态网络后,将源网络与这些静态网络逐个进行匹配结果衡量,最终选取值最大的结果。

有了动态匹配衡量标准的定义,对于本文研究的问题,便是找到一个匹配$f$使得动态匹配衡量标准最大化,即
\begin{equation}\label{myworktwobjdefine}
    objf=\underset{f}{argmax}Q(f)
\end{equation}
其中的$Q$可以是上述衡量标准的任意一项。针对$objf$,是否可以用已有的静态网络匹配算法来寻找呢?

%@@@@@@@@@@@@@@@@@@@@@@@@@@@@@@@@@@@@@@@@@@@@@@@@@@
\section{静态匹配算法的缺陷}
根据新的动态PPI网络匹配定义以及衡量匹配的标准,是否可以将已有的静态网络匹配算法应用到动态PPI网络匹配这个问题上呢?答案是肯定的,但是缺点就是时间或者匹配效果的下降。

第一种方案是,单纯忽略$L$和$R$的信息,将$G_2(V_2,E_2,L,R)$转化成对应的静态PPI网络$G_2(V_2,E_2)$,然后利用静态PPI网络匹配算法去算得匹配$f$,当做最后的结果$objf$。但是这样做的问题在于,直接把动态PPI网络的一条边当做静态网络边,意味着这条边在所有时刻都是活跃的,但是实际上这条边只在某些时刻才是处于活跃状态的,因此既有的匹配算法无论在相似度的估计上,还是匹配生成上,都会被这条看似永久活跃,其实只在某些时候活跃的边所“迷惑”,产生不精准的相似度估计或者匹配方案。

第二种方案,便是将动态PPI网络$G_2$拆分成若干张静态PPI网络,然后逐个进行匹配,从中选取拥有最好匹配效果的匹配方案。这种方法的好处在于,匹配算法确实是跑在了真实的网络上,但是缺点在于时间,假设动态PPI网络的时间跨度非常大,那么这种方法会让匹配算法非常耗时(从本来的只要匹配一次变成匹配$N$次),这对于一些已经比较耗时的匹配算法来说,无疑是不可接受的。

因此,我们希望得到一个不但时间消耗不大,且依旧能够得到较优匹配效果的算法。为此,本文提出了SGOPT(SeGment tree OPTimization)算法,不仅能够在既有静态匹配算法的基础上提高匹配效果,时间上也不会像逐个匹配一般地耗时,可以说是一种折中的算法,最关键的是,它是基于动态PPI网络所提出的算法,不同于已有的静态匹配算法,是对动态PPI网络这个新环境下的新问题的一种创新性的尝试。
%@@@@@@@@@@@@@@@@@@@@@@@@@@@@@@@@@@@@@@@@@@@@
\section{SGOPT算法}
这一部分主要介绍SGOPT算法,也是本文主要工作所在。可以看到,在所有动态匹配衡量标准的定义中,分子都是$|f_i(E_1)|$,而这个分子的意义,就是在$i$时刻,源网络$G_1$在通过匹配$f$的情况下,$E_1$中仍然被保留(匹配)的边的集合。一个直观的想法,就是不断调整匹配$f$,使得

\begin{equation}\label{myworkmaxfidefine}    
\underset{i}{max}\{ |f_i(E_1)|\}
\end{equation}

最大化。而要计算公式\ref{myworkmaxfidefine}的值,需要考查每一时刻$i$下,$E_1$中被保留的边数,假设时间跨度为$T$,则需要$\mathcal{O}(|E_1|*T)$的时间才能完成。SGOPT算法做的第一步,就是将这个时间从$\mathcal{O}(|E_1|*T)$变成了$\mathcal{O}(|E_1|*log(T))$,也就是说,对于$E_1$中的一条边,计算它在所有时刻内是否被保留,从原本的时间复杂度$\mathcal{O}(T)$,变成了$\mathcal{O}(log(T))$。

SGOPT算法的第二步,则是在第一步时间优化的基础上,通过一种局部调整的策略,逐步匹配$f$,使得公式\ref{myworkmaxfidefine}最大化。

SGOPT算法的第一步,从时间上优化了计算匹配衡量标准的计算耗时,第二步则是从第一步的基础上,使得对动态PPI网络的匹配,可以从逐个匹配的劣势下,变为只对一个网络进行匹配。

%@@@@@@@@@@@@@@@@@@@@@@@@@@@@@@@@@@@@@@@@@@@@@@@
\subsection{从$\mathcal{O}(T)$到$\mathcal{O}(log(T))$的转化}
先考虑这样一个问题,假设有一个整数序列,长度为$T$,序列为

\begin{equation}\label{myworkseqdefine}    
[a_1,a_2,a_3,.....,a_T]
\end{equation}

其中$a_i$为序列中第$i$个整数。现在有三种操作。

第一种,把第$l$个到第$r$个($l\leq r$)的所有整数都加上一个共同的数字$d$,简称$ADD(l,r,d)$。

第二种,把第$l$个到第$r$个($l\leq r$)的所有整数都减去一个共同的数字$d$,简称$SUB(l,r,d)$。

第二种,求出第$l$个到第$r$个($l\leq r$)的所有整数中最大的那个数,简称$MAX(l,r)$。

那么如何快速地完成这些操作呢?答案是用线段树(segment tree)这种数据结构\cite{de2000computational}。线段树是一种非常适合维护区间上操作的数据结构,对于一个长度不超过$T$的区间,其本质是一颗高度不超过$\mathcal{O}(log(T))$满二叉树,线段树的空间复杂度为$\mathcal{O}(T)$,而对于上述的三种操作中的任意一种操作,线段树都能够在不超过树高$\mathcal{O}(log(T))$的时间复杂度内完成。

那么这个问题对于本文研究的问题有什么作用呢?答案是可以将计算公式\ref{myworkmaxfidefine}的过程转化成该问题,然后用线段树在$\mathcal{O}(log(T))$时间内解决。

考虑本文研究问题的一个部分匹配$fp$,不同于$f$的是,$G_1$中的点$v$在部分匹配$fp$下可能并没有对应的$G_2$中的匹配点,用$fp(v)=undefined$来表示。同时用$fp^{-1}(v)=undefined$来表示$G_2$中的点$v$并没有被$G_1$中的任何一个点所匹配。

假设在$fp$这个匹配下,我们已经知道了所有的$|fp_i(E_1)|$的值($fp_i(E_1)$即为在匹配fp的情况下,在$i$时刻时$E_1$中被保留的边的集合。),这样的值共有$T$个,构成了一个整数序列,定义为

\begin{equation}\label{myworkfseqdefine}    
SEQ(fp)=[|fp_1(E_1)|,|fp_2(E_1)|,.....,|fp_T(E_1)|]
\end{equation}

现在,我们需要往$fp$匹配中添加一对新的匹配$(u,v)(u\in V_1,v\in V_2,fp(u)=undefined,fp^{-1}(v)=undefined)$。且令新的匹配为$fn=fp\bigcup (u,v)$。定义

\begin{equation}\label{myworkfadddefine}    
SEQ(fn)=ADD\_MATCH(SEQ(fp),(u,v))
\end{equation}

来表示这样的一种加匹配$(u,v)$的操作,在$ADD\_MATCH$操作后,$SEQ(fp)$变成了$SEQ(fn)$。而$ADD\_MATCH$操作,就是将$fp$匹配下各个时刻的$E_1$保留的边数,变成了$fn$匹配下各个时刻的$E_1$保留的边数。

在添加了匹配$(u,v)$后,考查满足$x\in N(u),y\in N(v),fp(x)=y$的任意点对$(x,y)$。由定义可知,在$E_1$中,边$(u,x)$通过匹配$fn$被映射到了$E_2$中的$(v,y)$,而边$(v,y)$在$[L(v,y),R(v,y)]$这个时间区间内是处于活跃状态的,所以对于任意时刻$i\in [L(v,y),R(v,y)]$,$fn_i(E_1)$因为$(u,v)$匹配的加入,而在$fp_i(E_1)$的基础上,多了$(u,x)$这一条边。也就是说,$SEQ(fp)$这个序列在$[L(v,y),R(v,y)]$这个区间内的所有数值都加了1,这不就是对$SEQ(fp)$进行了ADD(L(v,y),R(v,y),1)操作么?

如果用一颗长度为$T$的线段树来维护$SEQ(fp)$,那么每次的$ADD\_MATCH$操作,都可以通过对该线段树进行若干次(由符合条件的u和v的邻居点对个数决定)$ADD$操作来实现从$SEQ(fp)$到$SEQ(fn)$的转化,因此,其时间复杂度就是$\mathcal{O}(log(T))$。而通过不断加匹配,从部分匹配变成单射的时候(即$G_1$中每个点都得到了匹配),这颗线段树所维护的,正是$SEQ(f)$,而$SEQ(f)$中的最大值,正是公式\ref{myworkmaxfidefine}所要求的值,可以通过对该线段树进行$MAX(1,T)$操作来得到。这样,我们就完成了从$\mathcal{O}(T)$到$\mathcal{O}(log(T))$的转化。

%@@@@@@@@@@@@@@@@@@@@@@@@@@@@@@@@@@@@@@@@@@@@@@@@@@
\subsection{局部调整策略}
在有了线段树这一利器后,我们已经做到了可以对任意部分匹配$fp$,随时维护$SEQ(fp)$,根据$ADD\_MATCH$操作,我们可以从一个空的匹配集合和一颗序列长度为$T$且数值全为0的线段树开始,不断加入新的匹配$(u,v)$,在$\mathcal{O}(log(T))$的时间内更新序列,直到再没有匹配能够加入为止,最终产生的结果就是全局匹配$f$。

因此,利用这个框架,SGOPT算法可以结合任意已有的静态PPI网络匹配算法,将它们的匹配$f$结果作为输入,可以在$\mathcal{O}(|E_1|*log(T))$时间内算出$SEQ(f)$,比之前$\mathcal{O}(|E_1|*T)$的方式快了很多。

而且,在这个框架下,我们不但可以加入匹配,也可以删除匹配,同样可以在$\mathcal{O}(log(T))$时间内维护$SEQ(fp)$,比原本$\mathcal{O}(T)$的时间快了许多。因此,在这个优化框架下,本文提出了一种能够在动态PPI网络环境下的局部调整策略,使得得到的匹配在公式\ref{myworkmaxfidefine}下最大化。

类似于$ADD\_MATCH$,定义$DEL\_MATCH(SEQ(fn),(u,v))$操作表示将匹配$(u,v),fp(u)=v$从匹配$fn$中删去,而得到的新的匹配为$fp=fn-{(u,v)}$。对于删除匹配$(u,v)$,考查满足$x\in N(u),y\in N(v),fn(x)=y$的任意点对$(x,y)$,进行$SUB(L(v,y),R(v,y),1)$操作,同样可以在$\mathcal{O}(log(T))$的时间复杂度内完成。

于是,对于一个匹配$fp$,无论是添加一对匹配$(u,v)$,还是删除一对匹配$(u,v)$,都可以在$\mathcal{O}(log(T))$内做到。

局部调整策略的伪代码可以见算法\ref{alg:1}。算法的总体思想是通过随机删除已有匹配的一部分,并且加入一部分别的匹配来调整当前匹配。

$\alpha$和$\beta$都是参数,分别表示迭代次数上限以及匹配删除比例。

第1行到第3行是初始化过程,对于任意匹配$f$,算法将该匹配$f$作为需要调整的初始匹配,用线段树来维护$SEQ$序列。$ADD\_MATCH(SEQ,f)$是一系列$ADD\_MATCH(SEQ,(u,v)),(u,v)\in f$操作的集合(下同)。

第5行到第20行是整个迭代的循环,迭代的次数有$\alpha$参数决定,整个循环是不断调整既有匹配集合$fp$的过程。

第6行到第11行,算法从$fp$中随机挑出若干个匹配,挑出的数目由参数$\beta$决定,并且把这些匹配从$fp$中用操作$DEL\_MATCH(SEQ,f\_delete)$删去。

第12行到第12行,算法构造了一个二分图$bg$,点集分别有$G_1$和$G_2$中未被匹配的点构成,而任意点对$(u,v)$直接间的权重,由函数$NS(fp,u,v)$决定。

\begin{equation}\label{myworknsdefine}
    NS(f,u,v)=|\{(x,y):x\in N(u),y\in N(v),f(x)=y\}|
\end{equation}
$NS(f,u,v)$的值本质上,就是在匹配$f$的情况下,如果加入匹配$(u,v)$,能够对$E_1$中保留的边数增加的一个数值的估计,直观上可以认为,$NS(f,u,v)$越大,匹配$(u,v)$的效果越好。

第13行到第19行,得到二分图$bg$的最大权重匹配,将对应的匹配加入得到新的$fp$,最后比较新的$fp$在调整前后的好坏,如果好于调整前的,则说明这轮迭代找到了一个更优的匹配,否则,进行下一轮的迭代。

值得说明的是,这个二分图的最大权重匹配,不包括权值为0的边(对既有匹配没有任何贡献效益),所以在这个调整过程中,会存在某些点没有得到匹配,所以这些点在下一轮迭代中,和会那些被删掉匹配的点放在一起,重新构建一个新的二分图。这就导致了该算法可能会出现的“换入换出”(swap-in,swap-out)效果。每一次迭代,算法会“换出”一部分匹配,同时,“换入”一部分之前被删掉的匹配。本文认为,这样的局部调整策略具有较好的效果。

此算法的时间复杂度为$\mathcal{O}(\alpha\beta log(T))$,和两个参数息息相关。过小的参数可能会导致匹配效果糟糕,过大的参数,则会导致算法非常耗时,因此,需要在两个参数之间作出权衡。

\begin{small}
\begin{algorithm}[!htb]
{
\caption{局部调整算法}
\label{alg:1}
    \begin{algorithmic}[1]
    \Require
    $G_1(V_1,E_1)$:源网络
    
    $G_2(V_2,E_2,L,R)$:目标网络
    
    $f$:当前匹配
    
    $\alpha$:参数$\alpha>0$
    
    $\beta$:参数$0<\beta<1$
    
    \Ensure
    $fp$: 新的匹配
    
    \State $fp \gets f$
    \State $SEQ \gets [0,0,...,0]$
    \State $SEQ \gets ADD\_MATCH(SEQ,f)$
    \State $iteration\_count \gets 0$
    \While{$iteration\_count<\alpha$}
        \State $iteration\_count\gets iteration\_count+1$
        \State $oldSEQ\gets oldSEQ$
        \State $oldfp\gets fp$
        \State $f\_delete\gets$从$fp$中随机挑选的$\beta*|V_1|$个匹配
        \State $SEQ\gets DEL\_MATCH(SEQ,f\_delete)$
        \State $fp \gets fp-f\_delete$
        \State $bg\gets WeightedBipartiteGraph(\{u:fp(u)=undefined\},\{v:fp^{-1}(v)=undefined\},NS(fp,u,v))$
        \State $M\gets MaximumWeightedBipartiteGraphMaching(bg)$
        \State $fp\gets fp\bigcup M$
        \State $SEQ\gets ADD\_MATCH(SEQ,M)$
        \If{$MAX(SEQ)<MAX(oldSEQ)$}
            \State $SEQ\gets oldSEQ$
            \State $fp\gets oldfp$
        \EndIf
    \EndWhile
    \end{algorithmic}    
}
\end{algorithm}
\end{small}


