\renewcommand{\algorithmicrequire}{\textbf{Input:}}
\renewcommand{\algorithmicensure}{\textbf{Output:}}
\rhead{基于线段树优化的动态PPI网络匹配}
\chapter{基于线段树优化的动态PPI网络匹配}
%@@@@@@@@@@@@@@@@@@@@@@@@@@@@@@@@@@@@@@@@@@@@@
\section{动态PPI网络及动态PPI网络匹配}
由于本文要研究的问题是如何在动态PPI网络上作网络匹配,因此要先对动态PPI网络以及动态PPI网络匹配问题重新做出定义。

%@@@@@@@@@@@@@@@@@@@@@@@@@@@@@@@@@@@@@@@@@@@@@
\subsection{动态PPI网络定义}
借鉴于\cite{zhang2016method}关于动态PPI网络的定义,本文基于新的问题重新定义了动态PPI网络的概念。

一个动态PPI网络定义为五元组$G=(V,E,L,R,T)$,其中$V$是点集,$E$是边集,对于边集中的每一条边表示为$(u,v)$,其中$u$和$v$是在\textbf{某一时刻}具有相互作用的蛋白质点对(如果$u$和$v$在任意时刻都没有产生相互作用,则不存在$(u,v)$这条边)。$L$和$R$是关于每条边的一个函数,其中$L(u,v)$为边$(u,v)$处于活跃状态的开始时刻,$R(u,v)$为边$(u,v)$处于活跃状态的结束时刻。$T$是时间序列的跨度。$1\leq L(u,v)\leq R(u,v)\leq T$。为了简化问题以及算法的说明,这里假设动态PPI网络中每条边的活跃时刻是一个连续的时间区间。对于多个活跃区间的情况,本文的问题定义和算法也是同样适用的。定义

\begin{equation}\label{myworkactivedefine}
Act_i(u,v)=\begin{cases}
1 & \text{  } L(u,v)\leq i\leq R(u,v), (u,v)\in E\\ 
0 & \text{  其他}  
\end{cases}
\end{equation}

用来表示边$(u,v)$在$i$时刻是否处于活跃状态。定义$Act_i(E)=\{(u,v):(u,v)\in E,Act_i(u,v)=1\}$表示$i$时刻下边集E中活跃的边组成的集合。那么一个动态PPI网络,可以看成是T个静态PPI网络组成的集合$\{G_i(V_i,E_i):1\leq i\leq T\}$,$G_i$是$i$时刻下的静态PPI网络,其中$V_i=V$是该静态PPI网络的点集,边集$E_i$即$Act_i(E)$。

值得注意的是,在\cite{zhang2016method}中,动态PPI网络的定义与本文的定义稍有不同,前者的动态PPI网络中,一条边在某个时刻是否活跃是一个概率,而本文定义的动态PPI网络,在某一时刻,边$(u,v)$要么活跃,要么不活跃,去掉了概率这个因素。

%@@@@@@@@@@@@@@@@@@@@@@@@@@@@@@@@@@@@@@@@@@@
\subsection{动态PPI网络构造}
本文主要用了\cite{zhang2016method}中的三-西格玛阈值方法,对静态PPI网络进行了动态化的转化,由于\cite{zhang2016method}中构造的动态PPI网络,其每条边是否活跃是一个概率值,在本文的工作中,设定阈值0.5,即当边处于活跃状态的概率大于$0.5$时,本文就认为边就是处于活跃状态的,否则就是不活跃状态。详细的动态PPI网络构造方法可以参见相关工作。

%@@@@@@@@@@@@@@@@@@@@@@@@@@@@@@@@@@@@@@@@@@
\subsection{动态PPI网络匹配的问题定义}
本文研究的问题主要是一个静态PPI网络与一个动态PPI网络的匹配,两个动态PPI网络之间的匹配留作未来的研究工作。

有一个静态PPI网络$G_1=(V_1,E_1)$和一个动态PPI网络$G_2=(V_2,E_2,L,R,T)$,不失一般性的,$|V_1|\leq |V_2|$。定义\textbf{动态PPI网络的匹配}$f:V_1\rightarrow V_2$为点集$V_1$到点集$V_2$的一个单射。可以看到在有动态PPI网络的情况下,匹配的概念依然是和静态PPI网络时是一样的。

在一个动态PPI网络匹配$f$下,对于源网络$G_1$中的任意一条边$(u,v)$,如果在时刻$i$满足$Act_i(f(u),f(v))=1$,则称边$(u,v)$在\textbf{时刻}$\mathbf{i}$是被保留(匹配)的。

同样的,在匹配$f$下,定义$$f_i(E_1)=\{(u,v):(u,v)\in E_1,\text{且}(u,v)\text{在$i$时刻是被保留的}\}$$为源网络边集$E_1$在$i$时刻被保留的边的集合。

在动态PPI网络的环境下,所有的匹配衡量标准都需要重新定义,我们称这些衡量标准为\textbf{动态匹配衡量标准}(dynamic alignment quality measure)。分别有
\begin{equation}\label{myworkecdefine}
    EC(f)=\underset{i}{max}\frac{\left | f_i(E_1) \right |}{\left | E_1 \right |}
\end{equation}
衡量了动态PPI网络环境下,源网络$E_1$边集中,\textbf{所有时刻下},被保留的边的数量占自身边总数的比例的\textbf{最大值}。

\begin{equation}\label{myworkicsdefine}
    ICS(f)=\underset{i}{max}\frac{\left | f_i(E_1) \right |}{\left |Act_i(E_2(G_2[f(V_1)]))\right |}
\end{equation}

衡量了动态PPI网络环境下,目标网络由点集$f(V_1)$的导出子图在所有时刻下的活跃边中,被匹配到的边的数量占活跃边的总数量的比例的最大值。

\begin{equation}\label{myworks3define}
S^{3}(f)=\underset{i}{max}\frac{\left | f_i(E_1) \right |}{\left | E_1 \right |+\left |Act_i(E_2(G_2[f(V_1)])) \right |-\left | f_i(E_1) \right |}
\end{equation}
同时考虑了源网络及目标网络。

\begin{equation}\label{myworktwecdefine}
    TWEC(f)=\frac{EC(f)+ICS(f)}{2}
\end{equation}
则平衡了EC和ICS的指标。

可以看到,在动态PPI网络的环境下,所有的衡量标准,都得到了动态化的定义,其本质,就是在匹配$f$下,将动态PPI网络看成若干个静态PPI网络后,将源网络与这些静态PPI网络逐个进行匹配结果衡量,最终选取值最大的结果。

基于动态匹配衡量标准的定义,本文要研究的问题,便是找到一个动态PPI网络匹配$f$使得动态匹配衡量标准最大化,即
\begin{equation}\label{myworktwobjdefine}
    objf=\underset{f}{argmax}Q(f)
\end{equation}
其中的$Q$可以是上述衡量标准的任意一项。
%@@@@@@@@@@@@@@@@@@@@@@@@@@@@@@@@@@@@@@@@@@@@@@@@@@
\section{静态匹配算法的缺陷}
根据新的动态PPI网络匹配定义以及衡量匹配的标准,是否可以将已有的静态PPI网络匹配算法应用到动态PPI网络匹配这个问题上呢?答案是肯定的,但是缺点就是\textbf{时间}或者\textbf{匹配效果}的下降。本文提供两种解决方案。

%@@@@@@@@@@@@@@@@@@@@@@@@@@@@@@@@@@@@@@@@@
\subsection{方案一}
第一种方案,忽略动态PPI网络$G_2$的时间信息,也就是忽略$L,R,T$这三个部分的信息,直接将$G_2(V_2,E_2,L,R,T)$转化成对应的静态PPI网络$G_2(V_2,E_2)$。

然后,利用已有的静态PPI网络匹配算法对$G_1$和$G_2$进行匹配,令算得的匹配为$f$,并且将其当做最后的结果$objf$。

这样的方案的好处在于,能够利用已有的静态PPI网络匹配算法,解决动态PPI网络匹配的问题,但是,这样算得的匹配,在动态PPI网络的环境下,得到的各项匹配衡量指标,是否是最优的呢?答案显示是否定的,因为这样的做法是直接把动态PPI网络的一条边当做静态网络边,意味着这条边在所有时刻都是活跃的,但是实际上这条边只在某些时刻才是处于活跃状态的,因此,既有的静态匹配算法无论在相似度的估计上,还是在匹配点对生成上,都会被这条看似永久活跃,其实只在某些时刻活跃的边所“迷惑”,产生不精准的相似度估计或者匹配方案。
%@@@@@@@@@@@@@@@@@@@@@@@@@@@@@@@@@@@@@@@@@@@@@@@
\subsection{方案二}
第二种方案,便是将动态PPI网络$G_2$看成若干个静态PPI网络,然后利用已有的静态PPI网络匹配算法,逐个进行匹配并衡量匹配标准,从中选取拥有最好匹配效果的匹配方案作为最后的匹配结果$objf$。

这种方法的优点在于,算法确实是在对真实的PPI网络进行匹配,得到的匹配衡量指标具有较好的效果与真实性,然而这个方案的缺点也显而易见:非常消耗时间,假设动态PPI网络的时间跨度为$T$,而静态匹配算法匹配两个静态PPI网络的时间复杂度为$\mathcal{O}(S)$,那么这种方案总的时间复杂度便是$\mathcal{O}(S*T)$,相对于方案一来说,从本来的只要匹配一次变成了需要匹配$T$次,这对于一些已有的比较耗时的静态匹配算法来说,无疑是不可接受的。

总体上看,方案一时间快但效果差,而方案二效果好但十分耗时,是否能够有一个折中的方案呢?我们希望得到一个不但时间消耗不大,且依旧能够得到较优匹配效果的算法。

为此,本文提出了SGOPT(SeGment tree OPTimization)算法,不仅能够在既有静态匹配算法的基础上提高匹配效果,时间上也不会像逐个匹配一般地耗时,可以说是一种折中的算法,最关键的是,它是基于动态PPI网络所提出的算法,不同于已有的静态匹配算法,是对动态PPI网络这个新环境下的新问题的一种创新性的尝试。
%@@@@@@@@@@@@@@@@@@@@@@@@@@@@@@@@@@@@@@@@@@@@
\section{SGOPT算法}
这一部分主要介绍SGOPT算法,也是本文主要工作所在。可以看到,在所有动态匹配衡量标准的定义中,分子都是$|f_i(E_1)|$,而这个分子的意义,就是在$i$时刻,源网络$G_1$在匹配$f$下,$E_1$中仍然被保留(匹配)的边的数量。所以,一个直观的想法,就是不断调整匹配$f$,使得

\begin{equation}\label{myworkmaxfidefine}    
\underset{i}{max}\{ |f_i(E_1)|\}
\end{equation}

最大化。要计算公式\ref{myworkmaxfidefine}的值,需要考查每一时刻$i$下,$E_1$中被保留的边数,假设时间跨度为$T$,则暴力需要$\mathcal{O}(|E_1|*T)$的时间才能完成公式\ref{myworkmaxfidefine}的计算,计算方法为,查看$E_1$中的每一条边$(u,v)$,然后计算每个时刻下i,$Act_i(f(u),f(v))$的值,如果为1,则说明$(u,v)$在该时刻下是被保留的边,这个值可以在$\mathcal{O}(1)$时间内完成,所以完成所有$E_1$中的边的检验的总时间便是$\mathcal{O}(|E_1|*T)$。

而SGOPT算法要做的第一步,就是将这个时间从$\mathcal{O}(|E_1|*T)$变成了$\mathcal{O}(|E_1|*log(T))$,也就是说,对于$E_1$中的每一条边,计算它在每个时刻内是否被保留的信息,从原本的时间复杂度$\mathcal{O}(T)$,变成了$\mathcal{O}(log(T))$。

SGOPT算法的第二步,则是利用第一步可以高效计算公式\ref{myworkmaxfidefine}的基础上,通过一种局部调整当前匹配$f$的启发式策略,同时维护公式\ref{myworkmaxfidefine}的值,最终得到一个使得公式\ref{myworkmaxfidefine}的值最大化的匹配$f$。


%@@@@@@@@@@@@@@@@@@@@@@@@@@@@@@@@@@@@@@@@@@@@@@@
\subsection{从$\mathcal{O}(T)$到$\mathcal{O}(log(T))$的转化}
先考虑这样一个问题,假设有一个整数序列$a$,长度为$T$,序列$a$为

\begin{equation}\label{myworkseqdefine}    
a=[a_1,a_2,a_3,.....,a_T]
\end{equation}

其中$a_i$为序列中第$i$个整数。现在有三种操作。

第一种,把第$l$个到第$r$个($l\leq r$)的所有整数都加上一个共同的数字$d$,简称$ADD(l,r,d)$。

第二种,把第$l$个到第$r$个($l\leq r$)的所有整数都减去一个共同的数字$d$,简称$SUB(l,r,d)$。

第二种,求出第$l$个到第$r$个($l\leq r$)的所有整数中最大的那个数,简称$MAX(l,r)$。

如果用一个数组来维护序列$a$,则对于以上三个操作中的任意一个操作,可以在$\mathcal{O}(r-l)$的时间内暴力的维护序列中的每个整数值,但是其实有更好的方法。

那么如何快速地完成这些操作呢?答案便是使用线段树(segment tree)这种数据结构\cite{de2000computational}。线段树是一种非常适合维护区间上操作的数据结构,对于一个长度不超过$T$的区间,其本质是一颗高度不超过$\mathcal{O}(log(T))$满二叉树,线段树的空间复杂度为$\mathcal{O}(T)$,而对于上述的三种操作中的任意一种操作,线段树都能够在不超过树高$\mathcal{O}(log(T))$的时间复杂度内完成。

那么这个问题对于本文研究的问题有什么作用呢?答案是可以将计算公式\ref{myworkmaxfidefine}的过程转化成该问题,然后用线段树在$\mathcal{O}(log(T))$时间内解决。

考虑本文研究问题的一个部分匹配$fp$,不同于$f$的是,$G_1$中的点$v$在部分匹配$fp$下可能并没有对应的$G_2$中的匹配点,用$fp(v)=undefined$来表示。同时用$fp^{-1}(v)=undefined$来表示$G_2$中的点$v$并没有被$G_1$中的任何一个点所匹配。

假设在$fp$这个匹配下,我们已经知道了所有时刻$i$下,$|fp_i(E_1)|$的值($fp_i(E_1)$即为在匹配$fp$下,$E_1$在$i$时刻下被保留的边的集合。这样的值共有$T$个,构成了一个整数序列,定义为

\begin{equation}\label{myworkfseqdefine}    
SEQ(fp)=[|fp_1(E_1)|,|fp_2(E_1)|,.....,|fp_T(E_1)|]
\end{equation}

现在,如果我们要往匹配$fp$中添加一对新的匹配点对$(u,v)(u\in V_1,v\in V_2,fp(u)=undefined,fp^{-1}(v)=undefined)$,并且令新的匹配为$fn=fp\bigcup \{(u,v)\}$,那么可以将这种操作作如下定义

\begin{equation}\label{myworkfadddefine}    
SEQ(fn)=ADD\_MATCH(SEQ(fp),(u,v))
\end{equation}

那么,在$ADD\_MATCH$操作后,$SEQ(fp)$就变成了$SEQ(fn)$。而由$SEQ$的定义可知,$ADD\_MATCH$操作的本质,就是将匹配$fp$下各个时刻的$|fp_i(E_1)|$值,变成了匹配$fn$下各个时刻的$|fn_i(E_1)|$值。

现在证明可以利用线段树在$\mathcal{O}(log(T))$时间内完成$ADD\_MATCH$操作。

添加匹配点对$(u,v)$之后,考查满足$x\in N(u),y\in N(v),fn(x)=y$的任意点对$(x,y)$。由定义可知,在$E_1$中,边$(u,x)$通过匹配$fn$被映射到了$E_2$中的$(v,y)$,而边$(v,y)$在$[L(v,y),R(v,y)]$这个时间区间内是处于活跃状态的,所以对于任意时刻$i\in [L(v,y),R(v,y)]$,$fn_i(E_1)$因为$(u,v)$这一新的匹配点对的加入,而在$fp_i(E_1)$的基础上,多了$(u,x)$这一条被保留的边。也就是说,$SEQ(fp)$这个序列在$[L(v,y),R(v,y)]$这个区间内的所有数值都加了1,这不就是对$SEQ(fp)$进行了$ADD(L(v,y),R(v,y),1)$操作么?

如果用一颗长度为$T$的线段树来维护整数序列$SEQ(fp)$,那么每一次的$ADD\_MATCH$操作,都可以通过对该线段树进行若干次(由符合条件的u和v的邻居点对个数决定)$ADD$操作来实现从$SEQ(fp)$到$SEQ(fn)$的转化,因此,其总的时间复杂度便是$\mathcal{O}(log(T))$。

而通过不断加匹配点对,最终,部分匹配会变成一个完整的匹配$f$(即$G_1$中每个点都得到了匹配),此时,这颗线段树所维护的,正是$SEQ(f)$。而$SEQ(f)$中的最大值,正是公式\ref{myworkmaxfidefine}的值,这个值,则可以通过对该线段树进行$MAX(1,T)$操作来得到。

综上所述,对于一个当前的部分匹配,我们可以使用$ADD\_MATCH$操作不断往当前匹配中添加新的匹配点对,最终得到一个完整匹配。这之间所用的总时间是$\mathcal{O}(N*log(T))$,$N$为$ADD\_MATCH$的操作次数。

这样,我们就完成了从$\mathcal{O}(T)$到$\mathcal{O}(log(T))$的转化。

%@@@@@@@@@@@@@@@@@@@@@@@@@@@@@@@@@@@@@@@@@@@@@@@@@@
\subsection{启发式局部调整策略}
在有了线段树这一利器后,SGOPT算法缺少的,就是如何调整当前匹配的策略(添加或减少匹配点对)。

现在,我们已经做到了可以对任意部分匹配$fp$,用在$\mathcal{O}(log(T))$的时间维护$SEQ(fp)$。根据$ADD\_MATCH$操作,我们可以从一个空的匹配集合和一颗序列长度为$T$且数值全为0的线段树开始,不断加入新的匹配点对$(u,v)$,在$\mathcal{O}(log(T))$的时间内维护线段树,直到再没有匹配点对能够加入为止,最终产生的结果就是全局匹配$f$。

因此,利用这个框架,SGOPT算法可以结合任意已有的静态PPI网络匹配算法,将它们的匹配$f$结果作为输入,可以在$\mathcal{O}(|E_1|*log(T))$时间内算出$SEQ(f)$,比之前$\mathcal{O}(|E_1|*T)$的方式快了一个数量级。

而且,在这个框架下,我们不但可以加入匹配点对,也可以删除匹配点对,与加入的操作一样,删除操作同样可以在$\mathcal{O}(log(T))$时间内维护$SEQ(fp)$。

定义$DEL\_MATCH(SEQ(fn),(u,v))$操作表示将匹配$(u,v),fn(u)=v$从匹配$fn$中删去,而得到的新的匹配为$fp=fn-\{(u,v)\}$。即

\begin{equation}\label{myworkfdeletedefine}    
SEQ(fp)=DELETE\_MATCH(SEQ(fn),(u,v))
\end{equation}

对于将要删除的匹配点对$(u,v)$,考查满足$x\in N(u),y\in N(v),fn(x)=y$的任意点对$(x,y)$,进行$SUB(L(v,y),R(v,y),1)$操作便可以维护出$SEQ(fp)$,所以$DELETE\_MATCH$操作可以在$\mathcal{O}(log(T))$的时间复杂度内完成。

于是,对于一个匹配$fp$,无论是添加一对匹配点对$(u,v)$,还是删除一对匹配点对$(u,v)$,都可以在$\mathcal{O}(log(T))$内做到。在这个优化框架下,本文提出了一种在动态PPI网络环境下的启发式局部调整策略,策略的目标是尽量最大化公式\ref{myworkmaxfidefine}的值,以期能够获得效果较好的新匹配。

局部调整策略的伪代码可以见算法\ref{alg:1}。算法的总体思想是通过随机删除已有匹配的一部分,并且加入一部分别的匹配点对来调整当前匹配。

$\alpha$和$\beta$都是参数,分别表示\textbf{迭代次数上限}以及\textbf{匹配删除比例}。

第1行到第3行是初始化过程,对于任意匹配$f$,算法将该匹配$f$作为需要调整的初始匹配,用线段树来维护$SEQ$序列。$ADD\_MATCH(SEQ,f)$是一系列$ADD\_MATCH(SEQ,(u,v)),(u,v)\in f$操作的集合(下同)。

第5行到第20行是整个迭代的循环,迭代的次数由$\alpha$参数决定,整个循环是不断调整既有匹配$fp$的过程。

第6行到第11行,算法从$fp$中随机挑出若干个匹配点对,挑出的数目由参数$\beta$决定,并且把这些匹配点对从$fp$中用操作$DEL\_MATCH(SEQ,f\_delete)$删去。

第12行,算法构造了一个二分图$bg$,左右顶点集$X,Y$分别由$G_1$和$G_2$中未被匹配的点构成,而任意点对$(u,v)$间的二分图权值,则由函数$NS(fp,u,v)$决定。

\begin{equation}\label{myworknsdefine}
    NS(f,u,v)=|\{(x,y):x\in N(u),y\in N(v),f(x)=y\}|
\end{equation}
$NS(f,u,v)$的值本质上,就是在匹配$f$的情况下,如果加入匹配点对$(u,v)$,能够对$E_1$中保留的边数增加的一个数值的估计,直观上可以认为,$NS(f,u,v)$越大,添加匹配点对$(u,v)$的效果越好。

第13行到第19行,得到二分图$bg$的最大权值匹配集合$M$,将对应的匹配加入得到新的匹配$fp$,最后比较新的匹配$fp$在调整前后的效果差异,如果好于调整前的,则说明这轮迭代找到了一个更优的匹配,否则,抛弃当前的调整结果,进行下一轮的迭代。

值得说明的是,这个二分图的最大权值匹配,不包括权值为0的边(对既有匹配没有任何贡献效益的匹配点对)。所以在这个调整过程中后,会存在$G_1$中某些点没有得到匹配的情况。而这些点会在下一轮迭代中,和那些新的被删掉匹配的点放在一起,重新构建出一个二分图。可以看到这是一个“换入换出”(swap-in,swap-out)效果。每一次迭代,算法会“换出”一部分匹配点对,同时,“换入”一部分之前被删掉的匹配点对。本文认为,这样的局部调整策略具有较好的效果。

整个调整算法的时间复杂度为$\mathcal{O}(\alpha\beta log(T))$,和两个参数息息相关。过小的参数可能会导致匹配效果糟糕,过大的参数,则会导致算法非常耗时,因此,需要在两个参数之间作出权衡。

\begin{small}
\begin{algorithm}[!htb]
{
\caption{启发式局部调整算法}
\label{alg:1}
    \begin{algorithmic}[1]
    \Require
    $G_1(V_1,E_1)$:源网络
    
    $G_2(V_2,E_2,L,R,T)$:目标网络
    
    $f$:当前匹配
    
    $\alpha$:参数$\alpha>0$
    
    $\beta$:参数$0<\beta<1$
    
    \Ensure
    $fp$: 新的匹配
    
    \State $fp \gets f$
    \State $SEQ \gets [0,0,...,0]$
    \State $SEQ \gets ADD\_MATCH(SEQ,f)$
    \State $iteration\_count \gets 0$
    \While{$iteration\_count<\alpha$}
        \State $iteration\_count\gets iteration\_count+1$
        \State $oldSEQ\gets oldSEQ$
        \State $oldfp\gets fp$
        \State $f\_delete\gets$从$fp$中随机挑选的$\beta*|V_1|$个匹配点对
        \State $SEQ\gets DEL\_MATCH(SEQ,f\_delete)$
        \State $fp \gets fp-f\_delete$
        \State $bg\gets WeightedBipartiteGraph(\{u:fp(u)=undefined\},\{v:fp^{-1}(v)=undefined\},NS(fp,u,v))$
        \State $M\gets MaximumWeightedBipartiteGraphMaching(bg)$
        \State $fp\gets fp\bigcup M$
        \State $SEQ\gets ADD\_MATCH(SEQ,M)$
        \If{$MAX(SEQ)<MAX(oldSEQ)$}
            \State $SEQ\gets oldSEQ$
            \State $fp\gets oldfp$
        \EndIf
    \EndWhile
    \end{algorithmic}    
}
\end{algorithm}
\end{small}

如果动态PPI网络中的每条边的活跃区间,不止一个呢?

SGOPT算法对于不止一个活跃区间的情况,只要每次使用线段树进行操作的时候,从一个区间的操作,变成若干个区间操作叠加即可。增加的时间复杂度在于每条边的活跃区间个数。
