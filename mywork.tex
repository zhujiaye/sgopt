\renewcommand{\algorithmicrequire}{\textbf{Input:}}
\renewcommand{\algorithmicensure}{\textbf{Output:}}
\rhead{基于线段树优化的动态PPI网络比对}
\chapter{基于线段树优化的动态PPI网络比对}
%@@@@@@@@@@@@@@@@@@@@@@@@@@@@@@@@@@@@@@@@@@@@@
\section{动态PPI网络及动态PPI网络比对}
由于本文要研究的问题是如何在动态PPI网络上作网络比对,因此要先对动态PPI网络以及动态PPI网络比对问题重新做出定义。

%@@@@@@@@@@@@@@@@@@@@@@@@@@@@@@@@@@@@@@@@@@@@@
\subsection{动态PPI网络定义}
借鉴于\cite{zhang2016method}关于动态PPI网络的定义,本文基于新的问题重新定义了动态PPI网络的概念。

\begin{defn}[动态PPI网络]
\label{defndppi}
一个\textbf{动态PPI网络}为一个五元组$G=(V,E,L,R,T)$。$T$是一个整数,表示的是整个PPI网络的时间跨度。$V$是点集,$E$是边集,对于边集中的每一条边$(u,v)$,$u$和$v$是在\textbf{某一时刻}具有相互作用的蛋白质点对(如果$u$和$v$在任意时刻都没有产生相互作用,则不存在$(u,v)$这条边)。$L$和$R$是关于边集$E$的一个函数,其中$L(u,v)$为边$(u,v)$处于活跃状态的开始时刻,$R(u,v)$为边$(u,v)$处于活跃状态的结束时刻($1\leq L(u,v)\leq R(u,v)\leq T$)。
\end{defn}
为了简化问题以及算法的说明,这里假设动态PPI网络中每条边的活跃时刻是一个连续的时间区间。对于多个活跃区间的情况,本文的问题定义和算法也是同样适用的。

\begin{defn}[活跃边集合]
\label{defnactedge}
一个动态PPI网络$G$,在某时刻$i$下的\textbf{活跃边集合}为$E_i=\{(u,v):(u,v)\in E,L(u,v)\leq i\leq R(u,v)\}$。
\end{defn}

由\ref{defnactedge}可知,动态PPI网络$G$在每个时刻,活跃边的集合都是不一样的。这就是“动态”的意义所在。

\begin{defn}[分时网络]
\label{defnpartnetwork}
一个动态PPI网络$G(V,E,L,R,T)$,可以看成是若干个静态PPI网络组成的,表示为$G=[G_1,G_2,...,G_T]$。其中$G_i=(V,E_i)$是一个静态PPI网络,称为$G$在$i$时刻下的\textbf{分时网络},它是由点集$V$和$i$时刻下$G$的活跃边集合构成的网络。
\end{defn}

值得注意的是,在\cite{zhang2016method}中,动态PPI网络的定义与本文的定义稍有不同,前者的动态PPI网络中,一条边在某个时刻是否活跃是一个概率,而本文定义的动态PPI网络,在某一时刻,边$(u,v)$要么活跃,要么不活跃,去掉了概率这个因素。

%@@@@@@@@@@@@@@@@@@@@@@@@@@@@@@@@@@@@@@@@@@@
\subsection{动态PPI网络构造}
本文主要用了\cite{zhang2016method}中的三-西格玛阈值方法,对静态PPI网络进行了动态化的转化,由于\cite{zhang2016method}中构造的动态PPI网络,其每条边是否活跃是一个概率值,在本文的工作中,设定阈值0.5,即当边处于活跃状态的概率大于$0.5$时,本文就认为边就是处于活跃状态的,否则就是不活跃状态。详细的动态PPI网络构造方法可以参见相关工作。

%@@@@@@@@@@@@@@@@@@@@@@@@@@@@@@@@@@@@@@@@@@
\subsection{动态PPI网络比对}
本文研究的问题主要是一个静态PPI网络与一个动态PPI网络的比对,两个动态PPI网络之间的比对留作未来的研究工作。

\begin{defn}[动态PPI网络比对]
\label{defndppigna}
一个静态PPI网络$SG(SV,SE)$和一个动态PPI网络$DG(DV,DE,L,R,T)$(不失一般性,$|SV|\leq |DV|$)的\textbf{动态PPI网络比对}是一个从点集$SV$到$DV$的\textbf{映射}$f:SV\rightarrow DV,\forall v_1,v_2\in SV,f(v_1)\neq f(v_2)$。如果$SV$中的每个点都被映射了,则$f$是一个\textbf{单射}。
\end{defn}

可以看到在有动态PPI网络的情况下,网络比对的概念和静态PPI网络时是一致的。同理,匹配点,匹配点对,比对集合的概念也与静态PPI网络一致,这里就不再作定义,参见相关工作。

\begin{defn}[保留边,映射边]
\label{defncedge}
在一个动态PPI网络比对$f$下,对于源网络$SG$中的任意一条边$(u,v)\in SE$,如果满足$(f(u),f(v))\in DE_i$,则称边$(u,v)$在时刻$i$是被保留的,简称\textbf{保留边},并且称$i$时刻$DG$中对应的边$(f(u),f(v))$为\textbf{映射边}。$f_i(SE)=\{(f(u),f(v))\in DE_i:(u,v)\in SE\}$称为$SE$在时刻$i$时,在$DG$中的\textbf{映射边集合}。
\end{defn}

同静态PPI网络比对时类似的,在$i$时刻下,保留边和映射边是一一对应的。

在动态PPI网络的环境下,所有的比对衡量标准都需要重新定义,我们称这些衡量标准为\textbf{动态比对衡量标准}(dynamic alignment quality measure)。

\begin{defn}
\label{defndec}
动态PPI网络比对$f$的\textbf{DEC指标}为
$$DEC(f)=\underset{i}{max}\frac{\left | f_i(SE) \right |}{\left | SE \right |}$$
\end{defn}

DEC指标衡量了动态PPI网络环境下,边集$SE$在\textbf{所有时刻}下的保留边比例的\textbf{最大值}。

\begin{defn}
\label{defndics}
动态PPI网络比对$f$的\textbf{DICS指标}为$$DICS(f)=\underset{i}{max}\frac{\left | f_i(SE) \right |}{\left |DE_i(DG[f(SV)])\right |}$$
\end{defn}

DICS指标衡量了动态PPI网络环境下,$DG[f(SV)]$在\textbf{所有时刻}下的映射边比例的最大值。

\begin{defn}
\label{defnds3}
动态PPI网络比对$f$的\textbf{DS}$^3$\textbf{指标}为
$$DS^{3}(f)=\underset{i}{max}\frac{\left | f_i(SE) \right |}{\left | SE \right |+\left |DE_i(DG[f(SV)]) \right |-\left | f_i(SE) \right |}$$
\end{defn}


\begin{defn}
\label{defndtwec}
动态PPI网络比对$f$的\textbf{DTWEC指标}为$$DTWEC(f)=\frac{DEC(f)+DICS(f)}{2}$$
\end{defn}

可以看到,在动态PPI网络的环境下,所有的衡量标准,都得到了动态化的定义。

最后,给出本文要研究的动态PPI网络比对问题的形式化定义。

\begin{prob}[动态PPI网络比对问题]
\label{probdgna}
对于一个静态PPI网络$SG(SV,SE)$和一个动态PPI网络$DG(DV,DE,L,R,T)$,以及一个动态比对衡量标准$Q$,要求一个动态PPI网络比对$f$,使得在比对$f$下,$Q$的指标最大化。
\end{prob}
%@@@@@@@@@@@@@@@@@@@@@@@@@@@@@@@@@@@@@@@@@@@@@@@@@@
\section{静态比对算法的缺陷}
根据新的动态PPI网络比对定义以及衡量比对的标准,是否可以将已有的静态PPI网络比对算法应用到动态PPI网络比对这个问题上呢?答案是肯定的,但是缺点就是\textbf{时间}或者\textbf{比对效果}的下降。本文提供两种解决方案。

%@@@@@@@@@@@@@@@@@@@@@@@@@@@@@@@@@@@@@@@@@
\subsection{方案一}
第一种方案,忽略动态PPI网络$DG$的时间信息,也就是忽略$L,R,T$这三个部分的信息,直接将$DG(DV,DE,L,R,T)$转化成一个静态网络$DG(DV,DE)$。

然后,利用已有的静态PPI网络比对算法对$SG$和$DG$进行比对,得到对应的比对$f$,用它当做问题\ref{probdgna}的解。

这样的方案的好处在于,能够利用已有的静态PPI网络比对算法,解决动态PPI网络比对的问题,但是,这样算得的比对结果,在动态PPI网络的环境下,得到的各项比对衡量指标,是否是最优的呢?答案显然是否定的,因为这样的做法是直接把动态PPI网络的一条边当做静态网络边,意味着这条边在所有时刻都是活跃的,但是实际上这条边只在某些时刻才是处于活跃状态的,因此,既有的静态比对算法无论在相似度的估计上,还是在匹配点对生成上,都会被这条看似永久活跃,其实只在某些时刻活跃的边所“迷惑”,产生不精准的相似度估计或者比对方案。
%@@@@@@@@@@@@@@@@@@@@@@@@@@@@@@@@@@@@@@@@@@@@@@@
\subsection{方案二}
第二种方案,便是利用已有的静态PPI网络比对算法,将$SG$与动态PPI网络$DG$的所有分时网络$DG_i$一一作对比,并衡量静态比对指标,从中选取拥有最好比对效果的比对方案作为最后的比对$f$。

这种方法的优点在于,算法确实是在对真实的PPI网络进行比对,得到的比对衡量指标具有较好的效果与真实性,然而这个方案的缺点也显而易见:非常消耗时间,假设动态PPI网络的时间跨度为$T$,而静态比对算法比对两个静态PPI网络的时间复杂度为$\mathcal{O}(S)$,那么这种方案总的时间复杂度便是$\mathcal{O}(S*T)$,相对于方案一来说,从本来的只要比对一次变成了需要比对$T$次,这对于一些已有的比较耗时的静态比对算法来说,无疑是不可接受的。

总体上看,方案一时间快但效果差,而方案二效果好但十分耗时,是否能够有一个折中的方案呢?我们希望得到一个不但时间消耗不大,且依旧能够得到较优比对效果的算法。

为此,本文提出了SGOPT(SeGment tree OPTimization)算法,不仅能够在既有静态比对算法的基础上提高比对效果,时间上也不会像逐个比对一般的耗时,可以说是一种折中的算法,最关键的是,它是基于动态PPI网络所提出的算法,不同于已有的静态比对算法,是对动态PPI网络这个新环境下的新问题的一种创新性的尝试。
%@@@@@@@@@@@@@@@@@@@@@@@@@@@@@@@@@@@@@@@@@@@@
\section{SGOPT算法}
这一部分主要介绍SGOPT算法,也是本文主要工作所在。

由定义\ref{defndec},\ref{defndics},\ref{defnds3}可知,在所有动态比对衡量标准的定义中,分子都是$|f_i(SE)|$。

\begin{defn}[CE指标]
\label{defnce}
已知一个静态PPI网络$SG(SV,SE)$和一个动态PPI网络$DG(DV,DE,L,R,T)$,以及一个$SG$和$DG$之间的比对$f$,则比对$f$的\textbf{CE指标}为$$CE(f)=\underset{i}{max}\{ |f_i(SE)|\}$$。
\end{defn}

如何\textbf{最大化CE指标},便是SGOPT算法的目的。

首先,要计算CE指标的值,就需要知道所有$|f_i(SE)|$值,即每一时刻$i$下,$SG$中保留边集合的大小。


\begin{thm}
\label{thmfi}
已知一个静态PPI网络$SG(SV,SE)$和一个动态PPI网络$DG(DV,DE,L,R,T)$,以及一个$SG$和$DG$之间的比对$f$,计算$|f_i(SE)|$需要$\mathcal{O}(|SE|)$的时间。
\end{thm}
\begin{proof2}
由$f_i(SE)$的定义\ref{defncedge}可知,要计算$|f_i(SE)|$,考查$SE$中的每一条边在时刻$i$时是否被保留即可,总共有$\mathcal{O}(|SE|)$条边,而对于每一条边$(u,v)\in SE$,如果$(f(u),f(v))\in DE_i$,则说明$(u,v)$是保留边,而由$DE_i$的定义\ref{defnactedge},边$(f(u),f(v))$是否属于$DE_i$可在$\mathcal{O}(1)$的时间内判定。综上可知,计算$|f_i(SE)|$可在$\mathcal{O}(|SE|)$的时间内完成。
\end{proof2}
\begin{cor}
\label{corceforce}
已知一个静态PPI网络$SG(SV,SE)$和一个动态PPI网络$DG(DV,DE,L,R,T)$,以及一个$SG$和$DG$之间的比对$f$,则计算CE指标需要$\mathcal{O}(|SE|*T)$的时间。
\end{cor}
\begin{proof2}
由定理\ref{thmfi}可知,计算时刻$i$下的$f_i(SE)$需要$\mathcal{O}(|SE|)$的时间,这样的时刻共有$T$个,故需要$\mathcal{O}(|SE|*T)$的时间。
\end{proof2}
\begin{cor}
\label{corcesingle}
已知一个静态PPI网络$SG(SV,SE)$和一个动态PPI网络$DG(DV,DE,L,R,T)$,以及一个$SG$和$DG$之间的比对$f$,并且已知所有的$|f_i(SE)|$值,此时,如果比对$f$发生改变导致某条边$(u,v)\in SE$的映射关系发生变化,那么重新计算所有$|f_i(SE)|$的值需要$\mathcal{O}(T)$的时间。
\end{cor}
\begin{proof2}
如果只有边$(u,v)$的映射关系发生变化,只要单独重新检查$(u,v)$在每个时刻$i$下是否是保留边即可,总时间为$\mathcal{O}(T)$。
\end{proof2}
由推论\ref{corcesingle}可知,如果改变比对$f$,则需要在$\mathcal{O}(T)$的时间内重新维护CE指标。而SGOPT算法要做的第一步,就是将这个时间从$\mathcal{O}(T)$变成了$\mathcal{O}(log(T))$,大大降低了维护CE指标的时间。

SGOPT算法的第二步,则是设计了一种局部调整的启发式策略,来不断调整比对$f$,同时维护CE指标,最终得到一个CE指标最大化的比对$f$。
%@@@@@@@@@@@@@@@@@@@@@@@@@@@@@@@@@@@@@@@@@@@@@@@
\subsection{从$\mathcal{O}(T)$到$\mathcal{O}(log(T))$的转化}

在证明SGOPT算法可以将维护CE指标的时间从$\mathcal{O}(T)$降到$\mathcal{O}(log(T))$之前,先考虑另外一个问题。

\begin{prob}[整数序列问题]
\label{problist}
假设有一个整数序列$A$,长度为$T$,序列$A$为
\begin{equation*}   
A=[A_1,A_2,A_3,...,A_T]
\end{equation*}
其中$A_i$为序列中第$i$个整数。现在有三种操作。

第一种,给定$l,r,d(l\leq r,d>0)$,要求把序列$A$中的第$l$个到第$r$个整数都加上一个共同的正整数$d$,称作$ADD_A(l,r,d)$。

第二种,给定$l,r,d(l\leq r,d>0)$,要求把序列$A$中的第$l$个到第$r$个整数都加上一个共同的正整数$d$,称作$SUB_A(l,r,d)$。

第三种,给定$l,r(l\leq r)$,求出序列$A$中第$l$个到第$r$整数中最大的那个数,称作$MAX_A(l,r)$

问,完成这三种操作的时间复杂度分别是多少?
\end{prob}

如果用一个数组来表示序列$A$,则对于以上三个操作中的任意一个操作,可以在$\mathcal{O}(r-l)=\mathcal{O}(T)$的时间内暴力地更新每一个整数,但是其实有更好的方法。

\begin{lem}
\label{lemlistsg}
给定问题\ref{problist},可以在$\mathcal{O}(log(T))$时间内完成$ADD_A,SUB_A,MAX_A$三种操作。
\end{lem}
\begin{proof2}
要在$\mathcal{O}(log(T))$时间内完成三种操作,使用线段树(segment tree)即可。线段树\cite{de2000computational}是一种非常适合维护区间上操作的数据结构,对于一个长度不超过$T$的区间,其本质是一颗高度不超过$\mathcal{O}(log(T))$满二叉树,线段树的空间复杂度为$\mathcal{O}(T)$,而对于上述的三种操作中的任意一种操作,线段树都能够在不超过树高$\mathcal{O}(log(T))$的时间复杂度内完成。
\end{proof2}

那么这个问题的意义何在呢?答案便是可以将维护CE指标的问题转化成该问题,然后证明维护CE指标可以在$\mathcal{O}(log(T))$的时间内完成。

我们将推论\ref{corcesingle}推翻,重新定义维护CE指标的问题。

\begin{defn}[比对$f$的CE序列]
\label{defnfseq}
已知一个静态PPI网络$SG(SV,SE)$和一个动态PPI网络$DG(DV,DE,L,R,T)$,以及一个$SG$和$DG$之间的比对$f$,并且已知所有的$|f_i(SE)|$值,则整数序列$$SEQ(f)=[|f_1(SE)|,|f_2(SE)|,...,|f_T(SE)|]$$称为两个PPI网络在比对$f$下的\textbf{CE序列}。比对$f$的CE指标便是这个序列中的\textbf{最大整数},即$$CE(f)=MAX_{SEQ(f)}(1,T)$$
\end{defn}

\begin{prob}[维护CE指标的问题]
\label{probsingle}
已知一个静态PPI网络$SG(SV,SE)$和一个动态PPI网络$DG(DV,DE,L,R,T)$,以及一个$SG$和$DG$之间的比对$f$,并且已知$SEQ(f)$,此时,如果比对$f$发生改变导致某条边$(u,v)\in SE$的映射关系发生变化,那么重新计算$SEQ(f)$的时间复杂度是多少?
\end{prob}

下面证明可以将问题\ref{probsingle}转化为问题\ref{problist},从而在$\mathcal{O}(log(T))$时间内解决问题\ref{probsingle}。

\begin{thm}
\label{thmtrans}
问题\ref{probsingle}可以转化为问题\ref{problist},并且在$\mathcal{O}(log(T))$时间内解决。
\end{thm}
\begin{proof2}
由问题\ref{probsingle}的定义,假设变化前和变化后的比对分别是$fp$和$fn$,我们来看$SEQ(fn)$相比$SEQ(fp)$发生了什么变化。由于只有边$(u,v)$的映射关系发生了变化,我们可以将这个过程看成先去掉映射$u\rightarrow fp(u),v\rightarrow fp(v)$,然后添加新的映射$u\rightarrow fn(u),v\rightarrow fn(v)$。

我们先去掉两个映射关系$u\rightarrow fp(u),v\rightarrow fp(v)$。按照推论\ref{corcesingle},我们要检查$(u,v)$在每个时刻$i$时,映射边$(fp(u),fp(v))$和$DE_i$的关系。

由定义\ref{defnactedge},$(fp(u),fp(v))\in DE_i$当且仅当$(fp(u),fp(v))\in DE$且$L(fp(u),fp(v))\leq i\leq R(fp(u),fp(v))$。

情况一,$(fp(u),fp(v))\notin DE$,那么无论什么时刻,$(u,v)$都不可能成为一条保留边。所以去掉$u\rightarrow fp(u)$和$v\rightarrow fp(v)$映射关系后,所有的$|fp_i(SE)|$值不会发生改变。

情况二,$(fp(u),fp(v))\in DE$,那么在$[L(fp(u),fp(v)),R(fp(u),fp(v))]$这个时间区间内的任意时刻,$(u,v)$是一条保留边,去掉映射关系后,$\forall i\in [L(fp(u),fp(v)),R(fp(u),fp(v))],|fp_i(SE)|\leftarrow |fp_i(SE)|-1$,即$SUB_{SEQ(fp)}(L(fp(u),fp(v)),R(fp(u),fp(v)),1)$。

现在增加两个映射关系$u\rightarrow fn(u),v\rightarrow fn(v)$。也是分两种情况。

情况一,$(fn(u),fn(v))\notin DE$,添加映射关系后,所有的$|fp_i(SE)|$值不会发生改变,即$\forall i,|fn_i(SE)|=|fp_i(SE)|$。

情况二,$(fn(u),fn(v))\in DE$,添加映射关系后,$\forall i\in [L(fn(u),fn(v)),R(fn(u),fn(v))],|fn_i(SE)|\leftarrow |fp_i(SE)|+1$,即$ADD_{SEQ(fp)}(L(fn(u),fn(v)),R(fn(u),fn(v)),1)$。

可以看到,从$SEQ(fp)$到$SEQ(fn)$的变化,其实质,就是对一个整数序列进行ADD或SUB操作,而由定义\ref{defnfseq},要得到新比对$fn$的CE值标,对$SEQ(fn)$进行MAX操作即可。

这就完成了问题\ref{probsingle}到问题\ref{problist}的转化,而由引理\ref{lemlistsg}可知,用线段树可以在$\mathcal{O}(log(T))$时间内解决问题\ref{problist}。所以问题\ref{probsingle}可以在$\mathcal{O}(log(T))$的时间复杂度内解决。
\end{proof2}

由定理\ref{thmtrans}可以得到一个显然的推论。
\begin{cor}
\label{corcemany}
已知一个静态PPI网络$SG(SV,SE)$和一个动态PPI网络$DG(DV,DE,L,R,T)$,以及一个$SG$和$DG$之间的比对$f$,并且已知$SEQ(f)$,此时,如果比对$f$发生改变导致$SE$中的$N$条边的映射关系发生变化,那么重新计算$SEQ(f)$的时间复杂度为$\mathcal{O}(N*log(T))$
\end{cor}
%@@@@@@@@@@@@@@@@@@@@@@@@@@@@@@@@@@@@@@@@@@@@@@@@@@
\subsection{启发式局部调整策略}
在有了线段树这一利器后,SGOPT算法的第二步,就是通过不断调整比对$f$来最大化CE指标的过程。

\begin{defn}[比对单次调整]
\label{defnfsingle}
已知一个静态PPI网络$SG(SV,SE)$和一个动态PPI网络$DG(DV,DE,L,R,T)$,以及一个$SG$和$DG$之间的比对$f$。往比对集合$set(f)$加入一对匹配点对或删除$set(f)$中一对匹配点对,称为对比对$f$的\textbf{单次调整}。更形式化地,加入或删除一对匹配点对$(u,v)$用
$$fn=ADJ_{fp}((u,v))$$
来表示。其中$fp,fn$分别为调整前和调整后的比对。如果$u,v\notin set(fp)$,则是添加操作,如果$(u,v)\in set(fp)$则是删除操作。
\end{defn}

\begin{defn}[比对调整边集]
\label{defnadjedge}
已知一个静态PPI网络$SG(SV,SE)$和一个动态PPI网络$DG(DV,DE,L,R,T)$,以及一个$SG$和$DG$之间的比对$f$,进行$ADJ_f((u,v))$操作,那么集合$SE_f((u,v))=\{(u,x):x\in N(u)\}$表示$SE$中所有因为$ADJ_f((u,v))$操作而导致映射关系发生变化的边的集合,称为比对$f$关于$(u,v)$的\textbf{比对调整边集}。
\end{defn}

\begin{thm}[比对单次调整复杂度]
\label{thmfsingle}
已知一个静态PPI网络$SG(SV,SE)$和一个动态PPI网络$DG(DV,DE,L,R,T)$,以及一个$SG$和$DG$之间的比对$f$,并且已知$SEQ(f)$。如果进行$ADJ_f((u,v))$操作,那么重新计算$SEQ(f)$的时间复杂度为$\mathcal{O}(|N(u)|*log(T))$
\end{thm}
\begin{proof2}
由定义\ref{defnadjedge}可知,$ADJ_f((u,v))$操作后,$SE$中映射关系发生变化的边集为$SE_f((u,v))$,而|$SE_f((u,v))|=|N(u)|$,故由推论\ref{corcemany}可得,重新计算$SEQ(f)$的时间复杂度为$\mathcal{O}(|N(u)|*log(T))$
\end{proof2}

基于定理\ref{thmfsingle}本文提出了一种启发式的比对局部调整策略,通过不断进行比对的单次调整来最大化CE指标。

局部调整策略的伪代码可以见算法\ref{alg:1}。算法的总体思想是:每次迭代,随机删除当前比对集合的一部分,然后加入新的匹配点对来调整比对集合,如果得到的新比对CE指标较好,则保留,否则放弃这个新比对集合,继续下一轮迭代。

$\alpha$和$\beta$都是参数,分别表示\textbf{迭代次数上限}以及\textbf{匹配点对删除比例}。

第1行到第2行是初始化过程。

第3行到第14行是整个迭代的循环,迭代的次数由$\alpha$参数决定,整个循环是不断调整比对$fp$的过程。

第6行到第7行,算法从$set(fp)$中随机挑出若干个匹配点对,挑出的点对数目由参数$\beta$决定,并且把这些匹配点对从$set(fp)$中删去,调整比对$fp$。

第8行,算法构造了一个二分图$bg$,左右顶点集$X,Y$分别由$SG$和$DG$中未被匹配的点构成,而任意点对$(u,v)$间的二分图权值即$(u,v)$在比对$fp$下的贡献值$F_{fp}(u,v)$。

第9行,得到二分图$bg$的最大权值匹配集合$M$,要注意的是,这个二分图的最大权值匹配,不包括权值为0的边(对$fp$没有任何贡献效益的匹配点对)。

第10行,将$M$中的每一对匹配点对加入$set(fp)$来调整$fp$。

第11行到第13行,比较新的比对$fp$在调整前后的CE指标,如果好于调整前的,则说明这轮迭代找到了一个更优的比对,否则,抛弃当前的比对$fp$,进行下一轮的迭代。

整个调整算法的时间复杂度在于调整比对$fp$时的复杂度和作二分图最大权值匹配时的复杂度。

由定理\ref{thmfsingle},单次调整$fp$的复杂度为$\mathcal{O}(|N(u)|*log(T))$,而算法每一轮迭代的调整次数,由$\beta*|set(fp)|$决定,而因为算法的随机性,可以用$SG$的平均点度$|Avg(SG)|$来估计单次调整中的$|N(u)|$部分。调整$fp$的复杂度为$\mathcal{O}(\beta*|set(f)|*|Avg(SG)|*log(T))=\mathcal{O}(\beta*|SV|*|Avg(SG)|*log(T))$

对于二分图最大权值匹配,目前有不同的算法来解决这个问题,而各种算法各有各的好处。由于二分图匹配十分耗时,本文只用了贪心的方法来求二分图的最大权值匹配,时间复杂度为$\mathcal{O}(\beta*|set(f)|)=\mathcal{O}(\beta*|SV|)$

所以算法一轮迭代的时间复杂度为$\mathcal{O}(\beta*|SV|*|Avg(SG)|*log(T))$。迭代次数为$\alpha$,所以总的时间复杂度便是$\mathcal{O}(\alpha*\beta*|SV|*|Avg(SG)|*log(T))$。

可以看出,算法的运行时间与$\alpha,\beta$两个参数相关。过小的参数可能会导致糟糕的调整效果,过大的参数,则会导致算法非常耗时,因此,需要在两个参数之间作出权衡。

\begin{small}
\begin{algorithm}[!htb]
{
\caption{启发式局部调整算法}
\label{alg:1}
    \begin{algorithmic}[1]
    \Require
    $SG(SV,SE)$:源网络
    
    $DG(DG,DE,L,R,T)$:目标网络
    
    $f$:待调整比对
    
    $\alpha$:参数$\alpha>0$
    
    $\beta$:参数$0<\beta<1$
    
    \Ensure
    $fp$: 调整后的新比对
    
    \State $fp \gets f$
    \State $iteration\_count \gets 0$
    \While{$iteration\_count<\alpha$}
        \State $iteration\_count\gets iteration\_count+1$
        \State $oldfp\gets fp$
        \State $ds\gets$从$set(fp)$中随机挑选的$\beta*|set(fp)|$个匹配点对
        \State $\forall (u,v)\in ds,fp\gets ADJ_{fp}((u,v))$
        \State $bg\gets WeightedBipartiteGraph(\{u:u\notin set(fp)\},\{v:v\notin set(fp)\},\{F_{fp}(u,v)\})$
        \State $M\gets MaximumWeightedBipartiteGraphMaching(bg)$
        \State $\forall (u,v)\in M,fp\gets ADJ_{fp}((u,v))$
        \If{$CE(fp)<CE(oldfp)$}
            \State $fp\gets oldfp$
        \EndIf
    \EndWhile
    \end{algorithmic}    
}
\end{algorithm}
\end{small}

